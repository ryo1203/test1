\documentclass[dvipdfmx,twocolumn]{jsarticle}
\usepackage{mymacros}
\usepackage{amsmath,amssymb,amsthm}
\usepackage{tcolorbox}
\usepackage{mathrsfs}
\usepackage{tikz}
\title{微分積分学第一 中間試験}
\author{Jクラス}
\date{\today}
\begin{document}
\maketitle
\textbf{1.2}
(1)
\[\arcsin (- \frac{1}{2}) = - \frac{\pi}{3}\]
\[\arccos (- \frac{1}{2}) = \frac{\pi 5}{6}\]
\[\arctan (-1) = - \frac{\pi}{4}\]
(2)
$\arccos x = \arcsin \frac{3}{5}$ を満たす$x$を求める。

$\theta = \arcsin \frac{3}{5}$と置くと

$\arcsin$ と $\arccos$ の定義から
$\sin \theta = \frac{3}{5} , x = \cos \theta$となるので
$- \frac{\pi}{2} < \theta < \frac{\pi}{2}$から $\cos \theta  > 0$であることより
$\cos \theta = \sqrt{1 - \sin^2 \theta}$なので代入すると

$x = \sqrt{1 - (\frac{3}{5})^2} = \frac{4}{5}$となる。

(3)
\[
\frac{d}{dx} (\arcsin x + \arccos x) = \frac{1}{\sqrt{1 - x^2}} - \frac{1}{\sqrt{1 - x^2}} = 0
\]
なので $\arcsin x + \arccos x$は定数関数であることがわかる。

よって $x = 0$を代入すると
\[\arcsin 0 + \arccos 0 = 0 + \frac{\pi}{2}\]
となるため
$x$によらず
\[\arcsin x + \arccos x = \frac{\pi}{2}\]
が示された。

(4)
\[
\int \arctan x \, dx
\]
を求める。(ただし積分定数を $C$とする)

$x = \tan \theta$と置換すると逆関数であることを考えれば定義域は$- \frac{\pi}{2} < \theta < \frac{\pi}{2}$であり、
\[dx = \frac{d \theta}{\cos^2 \theta}\]
\[\arctan (\tan \theta) = \theta\]
となるので部分積分を用いて、 $\theta = \arctan x$を代入すれば
\begin{eqnarray*}
  \int \arctan x \, dx & = & \int \theta \frac{1}{\cos^2 \theta} \, d \theta \\
  & = & \theta \tan \theta - \int \tan \theta \,d \theta \\
  & = & \theta \tan \theta + \log |\cos \theta| + C \\
  & = & \arctan x \times x + \log |\cos (\arctan x)| + C \\
\end{eqnarray*}
ここで $\cos (\arctan x)$は定義域から正の値をとるので $\cos x = \frac{1}{\sqrt{1 + \tan^2 x}}$を用いれば
\begin{eqnarray*}
  \cos (\arctan x) & = & \frac{1}{\sqrt{1 + \tan^2 (\arctan x)}} \\
  & = & \frac{1}{\sqrt{1 + x^2}}
\end{eqnarray*}
となるため$\arctan x$の原始関数は
\[
\int \arctan x \, dx = x\arctan x - \frac{1}{2} \log (1 + x^2) + C
\]
である。 \\

\textbf{2.}
(1)
$y = (\tan x)^{\sin 2x}$を微分する。(ただし $0 < x < \frac{\pi}{2}$)
$\tan x > 0$から、両辺の $\log$をとると
\[
\log y = \sin 2x \log (\tan x)
\]
となるため対数微分法により両辺微分すると
\begin{eqnarray*}
\frac{y'}{y} & = & 2 \cos 2x \log (\tan x) + \sin 2x \frac{1}{\tan x} \frac{1}{\cos^2 x} \\
& = & 2 \cos 2x \log (\tan x) + 2 \sin x \cos x \times \frac{1}{\sin x \cos x} \\
\Leftrightarrow \\
y' & = & 2 (\tan x)^{\sin 2x} (\cos 2x \log (\tan x) + 1)
\end{eqnarray*}
となる。
\clearpage

(2)
$r,\theta$を $xy$平面上の極座標としたとき、ヤコビアン $\frac{\partial (x,y)}{\partial (r,\theta)}$を求める。

極座標であることから $x = r \cos \theta , y = r \sin \theta$なので
\[
\frac{\partial x}{\partial r} = \cos \theta \, , \, \frac{\partial y}{\partial \theta} = \sin \theta
\]
\[
\frac{\partial x}{\partial \theta} = - r \sin \theta \, , \, \frac{\partial y}{\partial \theta} = r \cos \theta
\]
であるのでヤコビアンは
\begin{eqnarray*}
\frac{\partial (x,y)}{\partial (r,\theta)} & = &
\mathrm{det}
\left(
\begin{array}{cc}
\frac{\partial x}{\partial r} & \frac{\partial y}{\partial \theta} \\
\frac{\partial x}{\partial \theta} & \frac{\partial y}{\partial \theta} \\
\end{array}
\right) \\
& = & r \cos^2 \theta - (- r \sin^2 \theta) \\
& = & r
\end{eqnarray*}
となる。

(3)
曲線 $C$が $x(t) = a \cos t , y(t) = b \sin t$とパラメーター表示されているとき $\frac{dy}{dx}$と $\frac{d^2y}{dx^2}$を求める。ただし $0 < t < \frac{\pi}{2}$であり $a$と$b$は正定数。
まず $\frac{dy}{dx}$は
\begin{eqnarray*}
  \frac{dy}{dx} & = & \frac{dy}{dt} \frac{1}{\frac{dx}{dt}} \\
  & = & b \cos t \frac{1}{- a \sin t} \\
  & = & - \frac{b}{a \tan t}
\end{eqnarray*}
である。 $\frac{d^2y}{dx^2}$は
\begin{eqnarray*}
\frac{d^2y}{dx^2} & = & \frac{d}{dx} \frac{dy}{dx} \\
& = & \frac{1}{\frac{dx}{dt}} \frac{d}{dt} \frac{dy}{dx} \\
& = & \frac{1}{- a \sin t} \left( - \frac{b}{a} \frac{-1}{\sin^2 t}  \right) \\
& = & - \frac{b}{a^2 \sin^3 x}
\end{eqnarray*}
である。 \\

\textbf{3.}
(1)
定積分 $\int_0^1 \frac{dx}{1 + x^3}$を求める。

$1 + x^3 = (1 + x)(1 - x + x^2)$より
$a,b,c$を用いて恒等式
\[
\frac{1}{(1 + x)(1 - x + x^2)} = \frac{a}{1 + x} + \frac{bx + c}{1 - x + x^2}
\]
を考えると
\[
a = \frac{1}{3} \, , \, b = - \frac{1}{3} \, , \, c = \frac{2}{3}
\]
となるため
\[\int_0^1 \frac{dx}{1 + x^3} =  \frac{1}{3} \int_0^1 \frac{1}{1 + x} + \frac{- x + 2}{1 - x + x^2} \, dx\]
 \[=  \frac{1}{3} \int_0^1 \frac{1}{1 + x} - \frac{1}{2} \frac{2x - 1}{1 - x + x^2} + \frac{\frac{3}{2}}{1 - x + x^2} \, dx \]
 \[=  \frac{1}{3} \left[ \log (1 + x) - \frac{1}{2} \log (1 - x + x^2) \right]_0^1 + \frac{1}{2} \int_0^1 \frac{dx}{(x^2 - x + \frac{1}{4}) + \frac{3}{4}}\]
 \[ = \frac{1}{3} \log 2 + \frac{1}{2} \int_0^1 \frac{dx}{(x - \frac{1}{2})^2 + \frac{3}{4}}\]
ここで上の式の第二項の積分を考える。
$x - \frac{1}{2} = t$と置くと $dx = dt$で、
\begin{table}[h]
  \begin{tabular}{c|r}
    $x$ & $0 \longrightarrow 1$ \\ \hline
    $t$ & $- \frac{1}{2} \longrightarrow \frac{1}{2}$
  \end{tabular}
\end{table}

であるから
\begin{eqnarray*}
\int_0^1 \frac{dx}{(x - \frac{1}{2})^2 + \frac{3}{4}} & = & \int_{- \frac{1}{2}}^{\frac{1}{2}} \frac{dt}{t^2 + \frac{3}{4}} \\
& = & \frac{4}{3} \int_{- \frac{1}{2}}^{\frac{1}{2}} \frac{dt}{\left( \frac{2t}{\sqrt{3}} \right)^2 + 1} \\
& = & \frac{4}{3} \left[ \frac{\sqrt{3}}{2} \arctan \left( \frac{2t}{\sqrt{3}} \right) \right]_{- \frac{1}{2}}^{\frac{1}{2}} \\
& = & \frac{2 \sqrt{3}}{3} \left\{ \arctan \left( \frac{1}{\sqrt{3}} \right) - \arctan \left( - \frac{1}{\sqrt{3}} \right) \right\} \\
& = & \frac{2 \sqrt{3} \pi}{9}
\end{eqnarray*}
となるため結果として
\[
\int_0^1 \frac{dx}{1 + x^3} = \frac{1}{3} \log 2 + \frac{\sqrt{3} \pi}{9}
\]
となる。

(2)
広義積分 $\int_1^\infty \frac{dx}{x (x + 1)}$を求める。
\begin{eqnarray*}
  \int_1^\infty \frac{dx}{x (x + 1)}
  & = & \int_1^\infty \frac{1}{x} - \frac{1}{x + 1} \, dx \\
  & = & \lim_{\beta \to \infty} \left[ \log x - \log (1 + x) \right]_1^\beta \\
  & = & \lim_{\beta \to \infty} \left\{ \log \left( \frac{\beta}{1 + \beta} \right) + \log 2 \right\} \\
  & = & \lim_{\beta \to \infty} \left\{ \log \left( \frac{1}{\frac{1}{\beta} + 1} \right) + \log 2 \right\} \\
  & = & \log 2 \\
\end{eqnarray*}

(3)
広義積分 $\int_1^\infty \frac{dx}{x^2 (x^2 + 1)}$を求める。
\begin{eqnarray*}
  \int_1^\infty \frac{dx}{x^2 (x^2 + 1)}
  & = & \int_1^\infty \frac{1}{x^2} - \frac{1}{x^2 + 1} \, dx \\
  & = & \lim_{\beta \to \infty} \left[ - x^{-1} - \arctan x \right]_1^\beta \\
  & = & \lim_{\beta \to \infty} \left\{ - \frac{1}{\beta} + 1 - \arctan \beta + \frac{\pi}{4} \right\}
\end{eqnarray*}
$\lim_{\beta \to \infty} \arctan \beta = 1$から
\[
\int_1^\infty \frac{dx}{x^2 (x^2 + 1)} = \frac{\pi}{4}
\]
\\

(4)
広義積分 $\int_1^\infty \frac{dx}{\sqrt{1 + x^4}}$が収束するか発散するかを求める。

$x > 1$において $1 + x^4 > x^4 > 1$よりルートをとると $\sqrt{1 + x^4} > x^2 > 1$が成り立つので
\begin{eqnarray*}
  \left| \frac{1}{\sqrt{1 + x^4}} \right| < \frac{1}{x^2}
\end{eqnarray*}
であり
\begin{eqnarray*}
  \int_1^\infty \frac{1}{x^2} \, dx
  & = & \lim_{\beta \to \infty} [ - x^{-1} ]_1^\beta \\
  & = & 1
\end{eqnarray*}
となり優関数の広義積分が収束するため $\int_1^\infty \frac{dx}{\sqrt{1 + x^4}}$も収束する。\\

\textbf{4.}
(1) $z = \arctan \frac{y}{x}$としたとき $z_{xx} + z_{yy}$を求める。
\begin{eqnarray*}
  z_x & = & \frac{1}{1 + \left( \frac{y}{x} \right)^2} \left( - \frac{y}{x^2} \right) \\
  & = & \frac{- y}{x^2 + y^2}
\end{eqnarray*}
なので
\[ z_{xx} = - y \frac{2x}{\left( x^2 + y^2 \right)^2} \]
また
\begin{eqnarray*}
  z_y & = & \frac{1}{1 + \left( \frac{y}{x} \right)^2} \left( \frac{1}{x} \right) \\
  & = & \frac{x}{x^2 + y^2} \\
\end{eqnarray*}
なので
\[z_{yy} = x \frac{2y}{x^2 + y^2}\]
よって
\[z_{xx} + z_{yy} = 0 \]
である。 \\

(2),(3)
左辺を計算するだけ。

飽きた。
\end{document}
